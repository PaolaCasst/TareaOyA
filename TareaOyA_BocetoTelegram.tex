\documentclass[12pt]{article} 

%Symbols
%\usepackage{txfonts}
\usepackage[spanish]{babel}
\usepackage{amsmath}
\usepackage{enumitem}
%\\usepackage{enumerate}
\usepackage[
  top=3cm,
  bottom=3cm,
  left=2.25cm,
  right=2.25cm,
  heightrounded]{geometry}

%Header-Footer
\usepackage{fancyhdr}
\cfoot{\thepage}
\lhead{Práctica 1: Medidas de rendimiento}
\rhead{\textsc{Organización y Arquitectura de Computadoras 2025-2}}
%\pagenumbering{gobble}

\pagestyle{fancyplain}

%tables
\setlength\tabcolsep{15pt}

%Margins
\usepackage{geometry}
\addtolength{\hoffset}{-0.2cm}
\addtolength{\textwidth}{0.4cm}
\addtolength{\voffset}{-0.5cm}
\addtolength{\textheight}{1cm}

%Comando para itemize y enumate.
\newcommand{\pl}[1]{\item \textbf{ #1 }}
\newcommand{\separador}{\begin{center}
    \Huge ***
    \textheight
\end{center}
}

\begin{document}
%\begin{titlepage}

%\include{Tareas/Plantilla}
%\end{titlepage}

\title{Practica 01 - Organización y Arquitectura de Computadoras}
\begin{center}
    \textbf{Castillo Chora Paola}\\
    \textbf{Martínez Jiménez José Guillermo}\\
    \textbf{Barrera Hernández Tomás}\\
    \textbf{García Serrano Jorge Eduardo}\\
    
\end{center}

\section{Especificaciones de las computadoras:}
\subsection{Integrante 1}
\begin{itemize}
    \pl{Nombre del intregrante:} Jorge Eduardo García Serrano.

    \pl{Fabricante y modelo de la computadora:} Sony Vaio modelo VPCF120FL.

    \pl{Tipo de Mother Board:} Sony VAIO (R1100Y6 BIOS).

    \pl{Fabricante, modelo, capacidad de la GPU, en caso de tenerla:} Fabricante Nvidia. Modelo NVIDIA GeForce GT 330M. 1GB de RAM (979MB usables), velocidad de núcleo 575MHz, velocidad de shader 1265 MHz, velocidad de memoria 1066MHz, tecnología 40nm.

    \pl{Fabricante, modelo, frecuencia, número de núcleos y arquitectura del procesador:} Fabricante Intel. Modelo Intel Core i7-740QM. Con 4 núcleos y con 8 subprocesos totales. Arquitectura de 64 bits (x64).

    \pl{Capacidad y tipo de memoria RAM y de caches de los procesadores:} Procesador con Caché L1 de 256 KB, L2 de 1 MB y L3 de 6 MB. Memorias RAM 2 slots, con la memoria del slot 1 de 4 GB 2RX8 PC3-12800S-11 DDR3 1600 CL11 marca ADATA, y con la memoria del slot 2 de 4 GB 2RX8 PC3-10600-9-10-F2 DDR3 marca HNNIX Korea 01.

    \pl{Capacidad, tipo y velocidad del disco duro:} Capacidad de 1 TB de almacenamiento, de tipo HDD magnético 2.5 pulgadas con sistema de archivos NTFS, velocidad promedio de lectura con escrituras secuenciales  10.405 ms, latencia máxima 90.900 ms, velocidad promedio de lectura con escrituras aleatorias 22.341 ms, lectura secuencial 64.0 73.82 MB/s, escritura secuencial 64.0 88.46 MB/s.

    \pl{Distribución del sistema operativo y versión del Kernel:} Windows 10 Pro 22H2 Build 19045. Versión de kernel 10.0.19045.5371 (x86\_64).
\end{itemize}

\subsection{Integrante 2}
\begin{itemize}
    \pl{Nombre del intregrante: Danna Paola Castillo Chora}

    \pl{Fabricante y modelo de la computadora: Apple, Macmini 7.1}

    \pl{Tipo de Mother Board: Utiliza una placa base diseñada específicamente por Apple, por lo que no tiene un modelo de motherboard comercial estándar}

    \pl{Fabricante, modelo, capacidad de la GPU, en caso de tenerla: Intel Iris (Gráficos Integrados)}

    \pl{Fabricante, modelo, frecuencia, número de núcleos y arquitectura del procesador: Intel(R) Core(TM) i5-4278U CPU @ 2.60GHz, Dual-Core Intel Core i5, 2.6 GHz, 2 nucleos}

    \pl{Capacidad y tipo de memoria RAM y de caches de los procesadores: 8 GB (2x4 GB), Caché L1 (instrucciones) 32 KB, Caché L1 (Datos) 32 KB, Caché L2 256KB, Caché L3 3MB}

    \pl{Capacidad, tipo y velocidad del disco duro: Disco Duro
1 TB (1000.0 GB) - SATA}

    \pl{Distribución del sistema operativo y versión del Kernel:macOS 12.7.6, 21.6.0}
\end{itemize}

\subsection{Integrante 3}
\begin{itemize}
    \pl{Nombre del intregrante: Tomás Barrera Hernández}
    
    \pl{Fabricante y modelo de la computadora: Hp,HP All-in-One 24-f0xx }
    
    \pl{Tipo de Mother Board:Motherboard Número de serie PGUUH0A8JDI1YC, versión 1000, modelo 8430, Fabricante Hp.}
    
    \pl{Fabricante, modelo, capacidad de la GPU, en caso de tenerla: Advanced Micro Devices, Inc [AMD/ATI], modelo Stoney[Radeon R2/R3/R4/R5 graphics], gráficos integrados.}
    
    \pl{Fabricante, modelo, frecuencia, número de núcleos y arquitectura del procesador:AMD A9-9425 AMD/ATI, modelo AMD A9-9425 RADEON R5 5 COMPUTE CORES, frecuencia de Mínima/Base: 1400 MHz (1.4 GHz) Máxima/Turbo: 3100 MHz (3.1 GHz) Velocidad Promedio (según la salida): 1409 MHz, número de núcleos 2 (Dual core), arquitectura de 64 bits.}
    
    \pl{Capacidad y tipo de memoria RAM y de caches de los procesadores:Capacidad de 8g, con dos ranuras 1 DIMM 0(CHANNEL A) Y 2 DIMM1(CHANNEL A) de 8 gb tipo DDR4, caché de los procesadores:
    	Caché L1 de datos (L1d):
    	Tamaño: 64 KiB por instancia
    	Número de instancias: 2
    	Total (si se sumaran todas): 2 × 64 KiB = 128 KiB
    	Caché L1 de instrucciones (L1i):
    	Tamaño: 128 KiB por instancia
    	Número de instancias: 2
    	Total (si se sumaran todas): 2 × 128 KiB = 256 KiB
    	Caché L2:
    	Tamaño: 2 MiB por instancia
    	Número de instancias: 2
    	Total (si se sumaran todas): 2 × 2 MiB = 4 MiB
    }
    
    \pl{Capacidad, tipo y velocidad del disco duro:Capacidad de disco de 1 Tb, disco duro mecánico (HDD) conectado vía SATA (rotacional), velocidad de escritura secuencial de 678 MB/s}
    
    \pl{Distribución del sistema operativo y versión del Kernel:Sistema operativo de Fedora Linux 40 (Workstation Edition), Versión 6.12.11, fedora 40 fc40, arquitectura x86-64 }
    \end{itemize}

\subsection{Integrante 4}
\begin{itemize}
    \pl{Nombre del intregrante: Martínez Jiménez José Guillermo}

    \pl{Fabricante y modelo de la computadora: ASUS, ASUS FX506LH}

    \pl{Tipo de Mother Board: ASUS FX506LH}

    \pl{Fabricante, modelo, capacidad de la GPU, en caso de tenerla: NVIDIA GeForce GTX 1650 + Intel UHD 4GB}

    \pl{Fabricante, modelo, frecuencia, número de núcleos y arquitectura del procesador:Intel Core i5-10300H @ 2.50GHz 4 nucleos x86-64}

    \pl{Capacidad y tipo de memoria RAM y de caches de los procesadores:1 x 8192MB 2933MHz Samsung M471A1K43DB1-CWE; cache L1 intruccione: 4 x 32kb, total 128 kb; cache L1 datos: 4 x 32kbs, total 128kb; cache L2: 4 x 256 kb, total 8mb}

    \pl{Capacidad, tipo y velocidad del disco duro:477GB Micron-2210-MTFDHBA512QFD + 1863GB WD Green SN350 2TB}

    \pl{Distribución del sistema operativo y versión del Kernel:Microsoft Windows 11 Home Single Language Build 26100 10.0.26100.3037 (x86-64)}
\end{itemize}

\newpage
\section{Tablas de resultados}

\begin{table}[!htb]
    \centering
    \begin{tabular}{|p{5cm}|p{9cm}|}
        \hline
        \textbf{Nombre de la prueba} & \textbf{Resultado de la prueba}\\
        \hline
        7Zip Compression &  Compresión promedio 11541 MIPS \newline
        Desviación de 1.10\% \newline
        Descompesión promedio 11227 MIPS \newline
        Desviación de 1.85\%
        En 4 minutos\\
        \hline
        Fhourstones &  Promedio de 7445.0 Kpos/sec \newline
        Desviación de 0.12\% \newline
        En 12 minutos\\
        \hline
        Xonotic (800x600 - Low) &  FPS promedio de 169.2188747 \newline
        FPS Min - 100, FPS Máx - 279 \newline
        Desviación de 0.48\% \newline
        En 13 minutos\\
        \hline
        Git &  Tiempo promedio de 126.663 seg. \newline
        Desviación de 1.13\% \newline
        En 15 minutos\\
        \hline
        BlogBench & Lectura promedio 3783 \newline
        Desviación de 9.33\% \newline
        Escritura promedio 70 \newline
        Desviación de 19.80\% \newline
        En 51 minutos\\
        \hline
        PHPBench & Puntuación promedio de 156712 \newline
        Desviación de 6.56\% \newline
        En 26 minutos\\
        \hline
        Unpacking The Linux Kernel &  Promedio de 63.977 seg. \newline
        Desviación de 19.77\% \newline
        En 23 minutos\\
        \hline
    \end{tabular}
    \caption{Resultado PC 1}
\end{table}

\begin{table}[!htb]
    \centering
    \begin{tabular}{|p{4cm}|p{10cm}|}
        \hline
        \textbf{Nombre de la prueba} & \textbf{Resultado de la prueba} \\
        \hline
        7Zip Compression & Compresión promedio de 10,958 MIPS, con una desviación estándar de 2.49\%. Descompresión de 8,796 MIPS con una desviación estándar de 1.74\%. Duración: 18 minutos. \\
        \hline
        Fhourstones & Promedio de 10,538.9 Kpos/sec, con una desviación estándar de 0.77\%. Rendimiento apenas inferior al promedio. Duración: 11 minutos. \\
        \hline
        Xonotic (800x600 - Low) & Promedio de 49.32 fps, con un mínimo de 32 fps y un máximo de 71 fps, con una desviación de 1.12\% y una duración de 15 minutos. \\
        \hline
        Team Fortress 2 & No se obtuvieron resultados debido a errores en la ejecución del benchmark. \\
        \hline
        Git & Promedio de 1461.126 segundos, con una desviación estándar de 208.14\% (altamente inconsistente). Duración: más de 4 horas. Anomalía en la sexta ejecución. \\
        \hline
        BlogBench & Promedio de lectura de 159,740, con una desviación estándar de 19.55\%. Promedio de escritura de 597, con una desviación estándar de 7.29\%. Duración total: 39 minutos. \\
        \hline
        Unpacking The Linux Kernel & Promedio de 71.85 segundos, con una desviación de 12.6\% y una duración de 6 minutos. \\
        \hline
        Vkpeak & No se obtuvo resultado, debido a la falta de compatibilidad con Vulkan en Intel Iris y macOS. \\
        \hline
        Phpbench & Promedio de 4,357,121 Score, con una desviación de 0.82\% y una duración de 4 minutos.\\
        \hline
    \end{tabular}
    \caption{Resultado PC 2}
    \label{tab:resultados_pc2}
\end{table}

\begin{table}[!htb]
    \centering
    \begin{tabular}{|p{5cm}|p{9cm}|}
        \hline
        \textbf{Nombre de la prueba} & \textbf{Resultado de la prueba}\\
        \hline
        7Zip Compression & Comprensión promedio de 5323 MIPS, \newline
        con una desviación estándar de 1.35\% \newline
        y una descompresión de 5943 MIPS \newline
        con una desviación estándar de 0.41\% con una duración de 18 minutos\\
        \hline
        Fhourstones & Promedio de 8239.3 Kpos/ses, \newline
        con una desviación de 1.92\%, \newline
        con un rendimiento bajo al promedio \newline
        y una duración de 10 minutos \\
        \hline
        Xonotic (800x600 - Low) & Promedio de 133.7280457 fps, \newline
        con un mínimo de 77 y un máximo de 224, \newline
        con una desviación de 0.34\% \newline
        y una duración de 18 minutos \\
        \hline
        Git & Promedio de 122.675 segundos, \newline
        con una desviación de 10.67\%, \newline
        con un resultado doble al estándar \newline
        y una duración de 8 minutos \\
        \hline
        BlogBench & Promedio de lectura de 429359.8, \newline
        con una desviación de 3.80\%, \newline
        un promedio de escritura de 468, \newline
        con una desviación del 0.12\% \newline
        y una duración total de 26 minutos \\
        \hline
        Unpacking The Linux Kernel & Promedio de 128.662 segundos, \newline
        con una desviación de 44.33\% \newline
        y una duración de 7 minutos \\
        \hline
        Vkpeak & Promedio de 75.86 GFLOPS, 75.9 GFLOPS, 4.78 GFLOPS, 4.76 GFLOPS, 15.39 GIOPS,15.27 GIOPS\newline
        con una desviación de 0.01\%,0.01\%, 0.00\%, 0.00\%, 0.04\%,0.04\%, \newline
        y una duración de 3 minutos \\
        \hline
        Phpbench & Promedio de 1 426932 Score, \newline
        con una desviación de 0.57\% \newline
        y una duración de 3 minutos \\
        \hline
    \end{tabular}
    \caption{Resultado PC 3}
\end{table}

\begin{table}[!htb]
    \centering
    \begin{tabular}{|p{5cm}|p{9cm}|}
        \hline
        \textbf{Nombre de la prueba} & \textbf{Resultado de la prueba}\\
        \hline
        7Zip Compression & Compresión: 23843 MIPS \newline 
        Descompresión: 21607 MIPS \\
        \hline
        Fhourstones & 11722 \\
        \hline
        Xonotic (800x600 - Low) &  375.43 fps\\
        \hline
        Git & 63.37 s\\
        \hline
        BlogBench & Lectura: 8582 \newline
        Escritura: 222 \\
        \hline
        PHPBench &  413051\\
        \hline
        Unpacking the linux kernel & 4.637 s\\
        \hline
    \end{tabular}
    \caption{Resultado PC 4}
\end{table}

\newpage
\section{Ejercicios}
\subsection{Integrante 1}

\begin{enumerate}[label=(\arabic{section}.\arabic{subsection}.\arabic{enumi})]
    \pl{Identifica cuáles de las pruebas miden el tiempo de respuesta y cuáles miden el rendimiento.}
    \begin{table}[htb]
        \centering
        \begin{tabular}{|c|c|}
        \hline
        Pruebas de tiempo de respuesta & Pruebas de rendimiento \\
        \hline
        7Zip  & BlogBench\\
        \hline
        Unpacking the Linux Kernel & Fhourstones \\
        \hline
        Git & Xonotic \\
        \hline
        & PhpBench \\
        \hline
        \end{tabular}
        \caption{ejercicio 3.1.1}
    \end{table}\par

    \pl{Usando la medida de tendencia central adecuada y tu reporte de resultados, calcula:}
    \begin{itemize}
        \pl{Medida de tiempo de respuesta:}(Indicar cuál medida se escogió y el resultado)\par
    Para 7Zip se usa la media aritmética:\par
        \begin{equation*}
            \frac{11397+11585+11640}{3}=11541
        \end{equation*}

        \vspace{0.5cm}
        Para Unpacking the Linux Kernel se usa la media aritmética:
        \[
        \frac{\begin{split}
            32.951 + 53.041 + 46.199 + 53.602 + 54.961 + 52.08 \\
            + \, 59.165 + 68.119 + 70.067 + 68.454 + 64.214 + 67.347 \\
            + \, 66.22 + 62.934 + 66.857 + 72.214 + 80.85 + 74.022 + 82.424 + 83.811
        \end{split}}{20} = 63.977
        \]
        \pl{Medida de rendimiento:} (Indicar cuál medida se escogió y el resultado)\par
        Para Fhourstones usamos la media armónica:
        \begin{equation*}
            \frac{3}{\frac{3}{7445.0}} = 7445.0
        \end{equation*}

        \vspace{0.5cm}
        Para BlogBench usamos la media aritmética:
        
        \textbf{Lectura:}
        \begin{equation*}
            \frac{2947 + 3826 + 3690 + 4111 + 3750 + 4167 + 3782 + 3820 + 3956}{9} = 3783
        \end{equation*}

        \textbf{Escritura:}
        \begin{equation*}
            \frac{54 + 75 + 80}{3} = 70
        \end{equation*}

        \vspace{0.5cm}
        Para Xonotic usamos la media aritmética:
        \begin{equation*}
            \frac{168.6185559 + 170.1518409 + 168.8862273}{3} = 169.2188747
        \end{equation*}

        \vspace{0.5cm}
        Para Git usamos la media aritmética:
        \begin{equation*}
            \frac{128.21 + 126.39 + 125.389}{3} = 126.663
        \end{equation*}

        \vspace{0.5cm}
        Para PhpBench usamos la media aritmética:
         \[
        \frac{\begin{split}
            151272 + 160059 + 168379 + 177297 + 169614 \\
            \, + 145203 + 153161 + 142647 + 152036 + 153266 + 152123 + 154490
        \end{split}}{12} = 156712
        \]
        \pl{Medida de tiempo de respuesta:}\par
        Para 7Zip se usa la media aritmética:\par
        \begin{equation*}
            \frac{11397+11585+11640}{3}=11541
        \end{equation*}

        \vspace{0.5cm}
        Para Unpacking the Linux Kernel se usa la media aritmética:
        \[
        \frac{\begin{split}
            32.951 + 53.041 + 46.199 + 53.602 + 54.961 + 52.08 \\
            + \, 59.165 + 68.119 + 70.067 + 68.454 + 64.214 + 67.347 \\
            + \, 66.22 + 62.934 + 66.857 + 72.214 + 80.85 + 74.022 + 82.424 + 83.811
        \end{split}}{20} = 63.977
        \]
        \pl{Medida de rendimiento:} \par
        Para Fhourstones usamos la media armónica:
        \begin{equation*}
            \frac{3}{\frac{3}{7445.0}} = 7445.0
        \end{equation*}

        \vspace{0.5cm}
        Para BlogBench usamos la media aritmética:
        
        \textbf{Lectura:}
        \begin{equation*}
            \frac{2947 + 3826 + 3690 + 4111 + 3750 + 4167 + 3782 + 3820 + 3956}{9} = 3783
        \end{equation*}

        \textbf{Escritura:}
        \begin{equation*}
            \frac{54 + 75 + 80}{3} = 70
        \end{equation*}

        \vspace{0.5cm}
        Para Xonotic usamos la media aritmética:
        \begin{equation*}
            \frac{168.6185559 + 170.1518409 + 168.8862273}{3} = 169.2188747
        \end{equation*}

        \vspace{0.5cm}
        Para Git usamos la media aritmética:
        \begin{equation*}
            \frac{128.21 + 126.39 + 125.389}{3} = 126.663
        \end{equation*}

        \vspace{0.5cm}
        Para PhpBench usamos la media aritmética:
         \[
        \frac{\begin{split}
            151272 + 160059 + 168379 + 177297 + 169614 \\
            \, + 145203 + 153161 + 142647 + 152036 + 153266 + 152123 + 154490
        \end{split}}{12} = 156712
        \]
    \end{itemize}
    
    \pl{Una vez que tengas los reportes de tus compañeros, cada alumno fijará su computadora como computadora de referencia, después calcula los tiempos normalizados y obtén la medida de tendencia central adecuada de cada una de las computadoras. Agrega cada tabla obtenida al reporte. Al final, el reporte deberá tener 4 tablas donde se usa cada equipo como computadora de referencia.}

    \begin{table}[htb]
        \centering
        \begin{tabular}{|c|c|c|c|c|}
        \hline
        \textbf{Nombre de la prueba} & \textbf{PC 1} & \textbf{PC 2} & \textbf{PC 3} & \textbf{PC 4}\\
        \hline
        7Zip Compression & & & & \\
        \hline
        Fhourstones & & & & \\
        \hline
        Xonotic (800x600 - Low) & & & &  \\
        \hline
        Git & & & &  \\
        \hline
        REDIS & & & &  \\
        \hline
        BlogBench & & & &  \\
        \hline
        Unpacking The Linux Kernel & & & & \\
        \hline
        \end{tabular}
        \caption{Usando la PC 1 como referencia (tiempo normalizado).}
    \end{table}

    \pl{Dada una prueba de rendimiento y otra de tiempo de respuesta, cada alumno deberá realizar el siguiente análisis: ¿Si pudieras cambiar una pieza de tu computadora para que la prueba se pudiera mejorar, qué cambiarías? ¿Cuánto costaría el cambio? ¿Cuánta sería la mejora que este cambio da?}
    
    \begin{itemize}
        \item De las pruebas de rendimiento escojo la de Fhourstones, y de las de tiempo de respuesta escojo la de 7Zip.

        \item Esto es complicado de mejorar con una sola pieza, diría que necesito mejorar al menos dos que serían la placa madre y el procesador. Esta laptop cuenta con el procesador Intel Core i7-740QM lanzada en Q3'10 (trimestre Julio-Septiembre de 2010), pero para mejorar mi procesador también necesito cambiar mi placa madre pues solo es compatible con procesadores de esos años y que ya están enlistados como \textit{Legacy} y descontinuados.

        \item Dado que varios procesadores revisados en Amazon y MercadoLibre rondan entre los \$1800 y los \$4000 MXN siendo más actuales, pero con la mayoría de ellos entre los \$2700 y los \$3500 MXN, diré que la mejora costaría al menos \$3000. Y para la placa madre solo pude encontrarlas en MercadoLibre (Amazon mostraba para PC de escritorio únicamente o tornillos), rondan un precio de \$1100 a \$6000 MXN, así que consideraré al menos \$2000 MXN. En total costaría al menos \$5000 MXN, lo que valdría casi lo mismo si compro una laptop usada con procesador de 2018-2022 que rondan en venta desde \$5800 a \$7000 MXN.

        \item Yendo por la opción de conseguir una laptop usada (pero conservando por el momento las tarjetas RAM actuales), diría que habría una mejora de al menos 120\% comparado con mis resultados actuales gracias a la velocidad del nuevo procesador y de los buses de la tarjeta madre actualizada. Si se considera mejorar la RAM de DD3 a DDR4, entonces habría una mejora del 200\%, puesto que estas tarjetas RAM instaladas también son usadas con un uso de al menos 3 años.\par
        Todo lo anterior en las mejoras de las pruebas es considerando que no son pruebas de disco, pues están utilizando el poder de procesamiento de la CPU.
    \end{itemize}

    \begin{itemize}
        \item De las pruebas de rendimiento escojo la de Fhourstones, y de las de tiempo de respuesta escojo la de 7Zip.

        \item Esto es complicado de mejorar con una sola pieza, diría que necesito mejorar al menos dos que serían la placa madre y el procesador. Esta laptop cuenta con el procesador Intel Core i7-740QM lanzada en Q3'10 (trimestre Julio-Septiembre de 2010), pero para mejorar mi procesador también necesito cambiar mi placa madre pues solo es compatible con procesadores de esos años y que ya están enlistados como \textit{Legacy} y descontinuados.

        \item Dado que varios procesadores revisados en Amazon y MercadoLibre rondan entre los \$1800 y los \$4000 MXN siendo más actuales, pero con la mayoría de ellos entre los \$2700 y los \$3500 MXN, diré que la mejora costaría al menos \$3000. Y para la placa madre solo pude encontrarlas en MercadoLibre (Amazon mostraba para PC de escritorio únicamente o tornillos), rondan un precio de \$1100 a \$6000 MXN, así que consideraré al menos \$2000 MXN. En total costaría al menos \$5000 MXN, lo que valdría casi lo mismo si compro una laptop usada con procesador de 2018-2022 que rondan en venta desde \$5800 a \$7000 MXN.

        \item Yendo por la opción de conseguir una laptop usada (pero conservando por el momento las tarjetas RAM actuales), diría que habría una mejora de al menos 120\% comparado con mis resultados actuales gracias a la velocidad del nuevo procesador y de los buses de la tarjeta madre actualizada. Si se considera mejorar la RAM de DD3 a DDR4, entonces habría una mejora del 200\%, puesto que estas tarjetas RAM instaladas también son usadas con un uso de al menos 3 años.\par
        Todo lo anterior en las mejoras de las pruebas es considerando que no son pruebas de disco, pues están utilizando el poder de procesamiento de la CPU.
    \end{itemize}

\end{enumerate}


\subsection{Integrante 2}

\begin{enumerate}[label=(\arabic{section}.\arabic{subsection}.\arabic{enumi})]
    \pl{Identifica cuáles de las pruebas miden el tiempo de respuesta y cuáles miden el rendimiento.}
    \begin{table}[htb]
        \centering
        \begin{tabular}{|c|c|}
        \hline
        Pruebas de tiempo de respuesta & Pruebas de rendimiento \\
        \hline
        7Zip Compression y Decompression & Fhourstones \\
        \hline
        Unpacking the Linux Kernel & Xonotic \\
        \hline
         & Git \\
        \hline
        & BlogBench \\
        \hline
        & Vkpeak \\
        \hline
        & Phpbench \\
        \hline
        \end{tabular}
        \caption{ejercicio 3.2.1}
    \end{table}\par

    \pl{Usando la medida de tendencia central adecuada y tu reporte de resultados, calcula:}
    \begin{itemize}
        \pl{Medida de tiempo de respuesta:}(Indicar cuál medida se escogió y el resultado)\par
    	Para la prueba de \textbf{compresión}:
La media aritmética se calcula de la siguiente manera:
\[
\frac{10,960 + 10,920 + 10,970 + 10,980}{4} = 10,958 \, 
\]

Para la prueba de \textbf{descompresión}:
La media aritmética se calcula de la siguiente manera:
\[
\frac{8,800 + 8,780 + 8,790 + 8,810}{4} = 8,796 \, 
\]

Para la prueba \textbf{Unpacking The Linux Kernel}, los valores de tiempo (en segundos) fueron:
\[
65.2, 72.1, 78.4, 69.8, 74.3, 70.9, 67.5, 73.6, 76.2, 71.1
\]
La media aritmética se calcula de la siguiente manera:
\[
\frac{65.2 + 72.1 + 78.4 + 69.8 + 74.3 + 70.9 + 67.5 + 73.6 + 76.2 + 71.1}{10} = 71.85 \, \text{segundos}
\]
        \pl{Medida de rendimiento:} (Indicar cuál medida se escogió y el resultado)\par
       
        \textbf{Fhourstones:} Para calcular la medida de rendimiento usamos la \textit{media armónica}:
\[
\text{Media armónica} = \frac{3}{\frac{1}{10622.6} + \frac{1}{10533.5} + \frac{1}{10460.6}} \approx 10536.2 \, \text{Kpos/sec}.
\]

Para la prueba de \textbf{Xonotic (800x600 - Low)}, usamos la media aritmética de los valores de fps:
\[
\frac{133.2123565 + 134.0232115 + 133.9485692}{3} = 133.7280457 \, \text{fps}.
\]
Duración: \textbf{15 minutos}.

Para la prueba de \textbf{Git}, usamos la media aritmética ponderada de los tiempos de ejecución (en segundos):
\[
\frac{120.144 + 130.532 + 148.672 + 143.233 + 121.799}{5} = 1461.126 \, \text{segundos}.
\]
Duración: \textbf{más de 4 horas}, con una desviación estándar muy alta y anomalía en la sexta ejecución.

Para la prueba de \textbf{BlogBench}, usamos la media aritmética para las mediciones de lectura y escritura:

\textbf{Lectura}:
\[
\frac{402312 + 441013 + 423612 + 431612 + 448250}{5} = 429359.8.
\]

\textbf{Escritura}:
\[
\frac{468 + 468 + 469}{3} = 468.33.
\]

Duración total: \textbf{39 minutos}.

Para la prueba de \textbf{Vkpeak}, usamos la media aritmética de los valores de rendimiento:

\textbf{fp32-scalar}:
\[
\frac{75.86 + 75.87 + 75.86}{3} = 75.86.
\]

\textbf{fp32-vec4}:
\[
\frac{75.89 + 75.90 + 75.90}{3} = 75.90.
\]

\textbf{fp64-scalar}:
\[
\frac{4.78 + 4.78 + 4.78}{3} = 4.78.
\]

\textbf{fp64-vec4}:
\[
\frac{4.76 + 4.76 + 4.76}{3} = 4.76.
\]

\textbf{int32-scalar}:
\[
\frac{15.39 + 15.40 + 15.39}{3} = 15.39.
\]

\textbf{int32-vec4}:
\[
\frac{15.27 + 15.27 + 15.28}{3} = 15.27.
\]
No se obtuvieron resultados debido a la falta de compatibilidad con Vulkan en Intel Iris y macOS.
    \end{itemize}

    \pl{Una vez que tengas los reportes de tus compañeros, cada alumno fijará su computadora como computadora de referencia, después calcula los tiempos normalizados y obtén la medida de tendencia central adecuada de cada una de las computadoras. Agrega cada tabla obtenida al reporte. Al final, el reporte deberá tener 4 tablas donde se usa cada equipo como computadora de referencia.}

     \begin{table}[ht]
\centering
\begin{tabular}{|l|c|c|c|c|}
\hline
\textbf{Nombre de la prueba} & \textbf{PC1} & \textbf{PC2} & \textbf{PC3} & \textbf{PC4} \\
\hline
\textbf{7Zip Compression (Compresión)} & 105.32 & 1 & 48.57 & 48.57 \\
\hline
\textbf{7Zip Compression (Descompresión)} & 127.88 & 1 &  67.6 & 67.6 \\
\hline
\textbf{Fhourstones} & 70.7 & 1 &  78.1 & 78.1 \\
\hline
\textbf{Xonotic (800x600 - Low)} & 342.68 & 1 &  270.6 & 270.6 \\
\hline
\textbf{Git} & 8.67 & 1 &  8.39 & 8.39 \\
\hline
\textbf{BlogBench (Lectura)} & 2.37 & 1 &  268.3 & 268.3 \\
\hline
\textbf{BlogBench (Escritura)} & 11.73 & 1 &  78.3 & 78.3 \\
\hline
\textbf{Unpacking The Linux Kernel} &  89.03 & 1  & 179.0 & 179.0 \\
\hline
\textbf{Phpbench} & 3.59 & 1 &  32.7 & 32.7 \\
\hline
\end{tabular}
\caption{Tiempos normalizados con PC2 como referencia}
\end{table}

    \pl{Dada una prueba de rendimiento y otra de tiempo de respuesta, cada alumno deberá realizar el siguiente análisis: ¿Si pudieras cambiar una pieza de tu computadora para que la prueba se pudiera mejorar, qué cambiarías? ¿Cuánto costaría el cambio? ¿Cuánta sería la mejora que este cambio da?}\\
\textbf{7Zip: Componente recomendado para mejorar el rendimiento: Procesador}

El \texttt{Intel Core i7-4770K}, que tiene 4 núcleos y 8 hilos. Este procesador podría duplicar el rendimiento en compresión y mejorar la descompresión, ya que estas tareas escalan bien con más núcleos.

El costo aproximado de este procesador es de \$1,500 - \$2,500 MXN.


\textbf{Fhourstones:}

El rendimiento de \texttt{Fhourstones} es de 10,538.9 Kpos/sec, pero el rendimiento es apenas inferior al promedio. Para mejorar este rendimiento, se recomienda cambiar el procesador por uno con más núcleos y mayor frecuencia, como el \texttt{Intel Core i7-4770K}, lo cual podría mejorar el rendimiento en tareas que involucren cálculo intensivo de hasta un 20-30.

Costo aproximado del cambio:  \$1,500 -  \$2,500 MXN.

\textbf{Xonotic (800x600 - Low):}

El rendimiento promedio es de 49.32 fps. Para mejorar la tasa de fotogramas por segundo, una posible mejora sería cambiar la tarjeta gráfica a un modelo con mejor soporte para altas tasas de cuadros, como una \texttt{NVIDIA GTX 1660}, lo que podría mejorar el rendimiento en fps hasta en un 40-50.

Costo aproximado del cambio:  \$3,000 -  \$5,000 MXN.

\textbf{Team Fortress 2:}

No se obtuvieron resultados debido a errores en la ejecución del benchmark. El componente más probable a revisar es la tarjeta gráfica o los controladores, ya que los errores podrían estar relacionados con incompatibilidades o cuellos de botella. Un cambio a una tarjeta gráfica más compatible y moderna, como la \texttt{NVIDIA GTX 1650}, podría mejorar la estabilidad y el rendimiento significativamente.

Costo aproximado del cambio:  \$2,500 -  \$4,500 MXN.

\textbf{Git:}

El rendimiento muestra una gran inconsistencia (deviación estándar de 208.14). Para mejorar la consistencia y reducir el tiempo de ejecución, un cambio a un SSD con mayor velocidad de lectura y escritura sería ideal, ya que las operaciones de Git se benefician mucho de una mayor velocidad de almacenamiento.

Costo aproximado del cambio:  \$1,000 -  \$2,500 MXN para un SSD de buena calidad.

\textbf{BlogBench:}

Para mejorar el rendimiento de lectura y escritura en \texttt{BlogBench}, se recomienda aumentar la RAM a 16GB, ya que las pruebas de escritura intensiva pueden beneficiarse de mayor capacidad de memoria. También se podría considerar el uso de un SSD de alta velocidad para reducir los tiempos de acceso a los datos.

Costo aproximado del cambio (RAM):  \$1,000 -  \$1,500 MXN.

Costo aproximado del cambio (SSD):  \$1,000 -  \$2,500 MXN.

\textbf{Unpacking The Linux Kernel:}

Para reducir los tiempos de desempaquetado, un cambio en el procesador a uno con más núcleos (por ejemplo, un \texttt{Intel Core i7}) podría mejorar el rendimiento en descompresión hasta en un 50, ya que este tipo de tareas se benefician de mayor paralelismo.

Costo aproximado del cambio:  \$1,500 -  \$2,500 MXN.

\textbf{Vkpeak:}

No se obtuvieron resultados debido a la falta de compatibilidad con Vulkan en el sistema. La actualización de la tarjeta gráfica a una que sea compatible con Vulkan, como la \texttt{NVIDIA RTX 3060}, podría permitir la ejecución de la prueba y mejorar el rendimiento en aplicaciones gráficas compatibles.

Costo aproximado del cambio:  \$5,000 -  \$8,000 MXN.

\textbf{Phpbench:}

El rendimiento es bastante consistente con una desviación estándar de 0.82, pero para mejorar el puntaje, un cambio al procesador \texttt{Intel Core i7-4770K} podría mejorar el rendimiento en tareas de cálculo, aumentando el puntaje entre un 15-30.
\end{enumerate}

\subsection{Integrante 3}

\begin{enumerate}[label=(\arabic{section}.\arabic{subsection}.\arabic{enumi})]
    \pl{Identifica cuáles de las pruebas miden el tiempo de respuesta y cuáles miden el rendimiento.}
    \begin{table}[htb]
        \centering
        \begin{tabular}{|c|c|}
        \hline
        Pruebas de tiempo de respuesta & Pruebas de rendimiento \\
        \hline
        7Zip Compression y Decompression & Fhourstones \\
        \hline
        Unpacking the Linux Kernel & Xonotic \\
        \hline
         & Git \\
        \hline
        & BlogBench \\
        \hline
        & Vkpeak \\
        \hline
        & Phpbench \\
        \hline
        \end{tabular}
    \end{table}\par

    \pl{Usando la medida de tendencia central adecuada y tu reporte de resultados, calcula:}
    \begin{itemize}
       \item \textbf{Medida de tiempo de respuesta:} \\
       Para la prueba 7-Zip Compression usamos la media aritmética:
       \[
       \text{Comprensión: } \frac{5360+5368+5240+5350}{4}=5323,\quad
       \]
       \[
       \text{Descompresión: } \frac{5952+5962+5916+5958}{4}=5943.
       \]
       Para la prueba Unpacking The Linux Kernel, usamos la media aritmética:
       \[
       \frac{
       	\begin{split}
       		83.213 + 23.127 + 164.781 + 114.976 + 105.271 + 44.262 \\
       		+\, 199.853 + 120.652 + 71.761 + 218.521 + 196.464 + 152.41 \\
       		+\, 128.269 + 114.311 + 124.175 + 196.541
       	\end{split}
       }{16} = 128.662.
       \]
       
       \item \textbf{Medida de rendimiento:} \\
       Para Fhourstones: Usamos la media armónica:
       \[
       \frac{3}{\frac{3}{8239.3}} = 8239.3\ \text{Kpos/sec}.
       \]
       Para Xonotic: Usamos la media aritmética:
       \[
       \frac{133.2123565+134.0232115+133.9485692}{3}=133.7280457\ \text{fps}.
       \]
       Para Git: Usamos la media aritmética ponderada y obtenemos:
       \[
       120.144\ \text{s}.
       \]
       Para Blogbench: Usamos la media aritmética:
       \[
       \text{Lectura: } \frac{402312+441013+423612+431612+448250}{5}=429359.8,\quad
       \]
       \[
       \text{Escritura: } \frac{468+468+469}{3}=468.33.
       \]
       	Para Vkpeak: Usamos la media aritmética:
       	\[
       	\text{fp32-scalar: } \frac{75.86+75.87+75.86}{3}=75.86 \quad
       	\]
       	\[
       	\text{fp32-vec4: } \frac{5.89+75.90+75.90}{3}=75.90 
       	\]
       	\[
       	\text{fp64-scalar: } \frac{4.78+4.78+4.78}{3}=4.78 \quad
       	\]
       	\[
       	\text{fp64-vec4: } \frac{4.76+4.76+4.76}{3}=4.76 \quad
       	\]
       	\[
       	\text{int32-scalar: } \frac{15.39+15.40+15.39}{3}=15.39 \quad
       	\]
       	\[
       	\text{int32-vec4: } \frac{15.27+15.27+15.28}{3}=15.27 \quad
       	\]
       \end{itemize}

    \pl{Una vez que tengas los reportes de tus compañeros, cada alumno fijará su computadora como computadora de referencia, después calcula los tiempos normalizados y obtén la medida de tendencia central adecuada de cada una de las computadoras. Agrega cada tabla obtenida al reporte. Al final, el reporte deberá tener 4 tablas donde se usa cada equipo como computadora de referencia.}

    \begin{table}[htb]
       \centering
        \begin{tabular}{|c|c|c|c|c|}
        \hline
        \textbf{Nombre de la prueba} & \textbf{PC 1} & \textbf{PC 2} & \textbf{PC 3} & \textbf{PC 4}\\
        \hline
        7Zip Compression & $\frac{11541}{5323}=2.1681$ & $\frac{23843}{5323}=4.4792$ & $\frac{5323}{5323}=1$ & $\frac{10958}{5323}=2.0586$ \\
        \hline
        7Zip Descompression& $\frac{11227}{5943}=1.8891$ & $\frac{21607}{5943}=3.6357$ & $\frac{5943}{5943}=1$ &  $\frac{8796}{5943}=1.4801$ \\
        \hline
        Fhourstones & $\frac{7445.0}{8239.3}=0.9036$ & $\frac{11722}{8239.3}=1.4227$ & $\frac{8239.3}{8239.3}=1$ & $\frac{10538.9}{8239.3}=1.2791$ \\
        \hline
        Xonotic (800x600 - Low) & $\frac{169.2189}{133.728}=1.2654$ & $\frac{375.43}{133.728}=2.8074$ & $\frac{133.728}{133.728}=1$ & $\frac{49.32}{133.728}=0.3688$ \\
        \hline
        Git & $\frac{126.663}{122.675}=1.0325$ & $\frac{63.37}{122.675}=0.5166$ & $\frac{122.675}{122.675}=1$ & $\frac{1461.126}{122.675}=11.9105$ \\
        \hline
        BlogBench (Lectura) & $\frac{3783}{429359.8}=0.0088$ & $\frac{8582}{429359.8}=0.0200$ & $\frac{429359.8}{429359.8}=1$ & $\frac{159740}{429359.8}=0.3720$ \\
        \hline
        BlogBench (Escritura) & $\frac{70}{468}=0.1496$ & $\frac{222}{468}=0.4744$ & $\frac{468}{468}$ & $\frac{597}{468}=1.2756$ \\
        \hline
        Unpacking The Linux Kernel & $\frac{63.977}{128.662}=0.4972$ & $\frac{4.637}{128.662}=0.0360$ & $\frac{128.662}{128.662}=1$ & $\frac{71.85}{128.662}=0.5584$ \\
        \hline
        Vkpeak & \textit{Sin resultados} & \textit{Sin resultados} & Sin resultados & \textit{Sin resultados} \\
        \hline
        Phpbrenchv & $\frac{156712}{426932}=0.3671$ & $\frac{413051}{426932}=0.9675$ & $\frac{426932}{426932}=1$ & $\frac{4357121}{426932}=10.2057$ \\
        \hline
        \end{tabular}
        \caption{Usando la PC 3 como referencia (tiempo normalizado).}
    \end{table}

    \begin{itemize}
    	\item \textbf{Dada una prueba de rendimiento y otra de tiempo de respuesta, cada alumno deberá realizar el siguiente análisis:} ¿Si pudieras cambiar una pieza de tu computadora para que la prueba se pudiera mejorar, qué cambiarías? ¿Cuánto costaría el cambio? ¿Cuánta sería la mejora que este cambio da? Para la prueba 7Zip:
    	
    	\item \textbf{7Zip:} Cambiar el Disco Duro (HDD) a una unidad de Estado Sólido (SSD), con el costo aproximado de \$700 MXN, teniendo un impacto de que al realizar el cambio se pueden mejorar los tiempos de arranque hasta en 30 segundos y reducir los tiempos de carga hasta en un 60\%.
    	
    	\item \textbf{Fhourstones:} Actualizar el CPU por un AMD Ryzen 5 5600G, con un costo aproximado de \$2,100 MXN, teniendo un impacto en el rendimiento, pues mejora el procesamiento de cálculos intensivos, con una mayor cantidad de núcleos e hilos reduciendo el tiempo de ejecución, mejorando un 40\% en cálculos de búsqueda como algoritmos alfa-beta.
    	
    	\item \textbf{Xonotic:} El componente a cambiar en este caso sería el GPU, pues tiene mayor impacto en los frames y benchmarks gráficos, si la cambiamos por una tarjeta gráfica de nivel medio como AMD Radeon RX 6600 con un costo de \$7,000-\$8,000 MXN, el rendimiento podría mejorar desde un 50\% hasta un 100\% de mejora en escenarios de juegos o bien pruebas gráficas.
    	
    	\item \textbf{Git:} De igual manera que con el test de 7Zip, lo mejor sería cambiar el Disco Duro (HDD) a una unidad de Estado Sólido (SSD), con el costo aproximado de \$700 MXN, teniendo un impacto al reducir el tiempo de acceso a archivos y en operaciones de lectura/escritura, así mismo como mejora del rendimiento en tareas de clonación y cambios en repositorios grandes, con una posible mejora del 30-50\% en la velocidad de ejecución.
    	
    	\item \textbf{Blogbench:} Cambiar el Disco Duro (HDD) a una unidad de Estado Sólido (SSD) es la mejor opción, con el costo aproximado de \$700 MXN, teniendo un impacto en la reducción de tiempos para acceder y mejorar la velocidad de lectura/escritura, mejorando en la lectura en un 50\% y en la escritura en un 80\%-90\%.
    	
    	\item \textbf{Unpacking:} Por última vez, cambiar el Disco Duro (HDD) a una unidad de Estado Sólido (SSD) es la mejor opción, con el costo aproximado de \$700 MXN, teniendo un impacto en la reducción de tiempos para la extracción de información, así como lectura y escritura de datos.
    	
    	\item \textbf{Vkpeak:} Cambiar el GPU pues tiene mayor impacto a la hora de ejecutar operaciones en FP32 y FP64 e int32, cambiandolo por una tarjeta gráfica de nivel medio como RTX 3060, con un costo aproximado de \$9,000 MXN,aumentando los valores de rendimiento, operaciones vectoriales y cálculos paralelos, con un mejor rendimiento
    	
    	\item \textbf{Phpbench:} De nuevo, sería un cambio de CPU, con un cambio a un procesador de gama media, como el AMD Ryzen 5 5600G, con un costo de  \$2,100 MXN, para que las tareas ejecutadas en el benchmark mejores en cálculos y procesamiento, así como un mayor rendimiento con mayor velocidad de reloj y con más núcleos/hijos acelerando la ejecución de scripts y tareas.
    	
    \end{itemize}

\end{enumerate}

\subsection{Integrante 4}

\begin{enumerate}[label=(\arabic{section}.\arabic{subsection}.\arabic{enumi})]
    \pl{Identifica cuáles de las pruebas miden el tiempo de respuesta y cuáles miden el rendimiento.}
    \begin{table}[htb]
        \centering
        \begin{tabular}{|p{6cm}|p{6cm}|}
        \hline
        Pruebas de tiempo de respuesta & Pruebas de rendimiento \\
        \hline
        Git & 7-zip compression \\
        \hline
        Unpacking the linux kernel & Fhourstones \\
        \hline
        & Xonotic (800x600 - Low)\\
        \hline
        & Blogbench\\
        \hline
        \end{tabular}
    \end{table}\par

    \pl{Usando la medida de tendencia central adecuada y tu reporte de resultados, calcula:}
    \begin{itemize}
        \item \textbf{Medida de tiempo de respuesta:} \\
       Para la prueba Git: Usamos la media aritmética:
       \[
       \text{Git:} = 63.37 \text{s}\quad
       \]
       Para la prueba Unpacking The Linux Kernel: Usamos la media aritmética:
       \[
       \text{Unpacking:} = 4.637 \text{s}\quad
       \]
       \item \textbf{Medida de rendimiento:} \\
       Para 7-zip compression: Usamos la media aritmetrica:
       \[
       \text{Compresion} = 23843\  \text{MIPS}\quad
       \]
       \[
       \text{Decompresión} = 21607\  \text{MIPS}
       \]
       Para Fhourstones: Usamos la media aritmétrica ponerada:
       \[
       \text{Fhourstones:}= 11722\  \text{Kpos/sec}.
       \]
       Para Xonotic: Usamos la media aritmética ponerada:
       \[
       \text{Xonotic:}=375.43\ \text{fps}.
       \]
       Para Blogbench: Usamos la media aritmétrica:
       \[
       \text{Lectura: } =8582\quad
       \]
       \[
       \text{Escritura: } =222
       \] 
       Para PHPBench: Usamos la media geométrica:
       \[
       \text{PHPBench:} = 413051
       \]
    \end{itemize}

    \pl{Una vez que tengas los reportes de tus compañeros, cada alumno fijará su computadora como computadora de referencia, después calcula los tiempos normalizados y obtén la medida de tendencia central adecuada de cada una de las computadoras. Agrega cada tabla obtenida al reporte. Al final, el reporte deberá tener 4 tablas donde se usa cada equipo como computadora de referencia.}

    \begin{table}[htb]
        \centering
        \begin{tabular}{|c|c|c|c|c|}
        \hline
        \textbf{Nombre de la prueba} & \textbf{PC 1} & \textbf{PC 2} & \textbf{PC 3} & \textbf{PC 4}\\
        \hline
        7Zip Compression & 0.5 & 0.43 & 0.24 & 1 \\
        \hline
        Fhourstones  & 0.6 & 0.89 & 0.7 & 1 \\
        \hline
        Xonotic (800x600 - Low) & 0.45 & 0.13 & 0.35 & 1 \\
        \hline
        Git & 1.99 & 23.05 & 1.93 & 1 \\
        \hline
        PHPBench & 0.37 & 10.54 & 1.03 & 1 \\
        \hline
        BlogBench & 0.37  & 7.07 & 10.26 & 1 \\
        \hline
        Unpacking The Linux Kernel & 13.81 & 15.51 & 27.78 & 1  \\
        \hline
        \end{tabular}
        \caption{Usando la PC 4 como referencia (tiempo normalizado).}
    \end{table}

    \pl{Dada una prueba de rendimiento y otra de tiempo de respuesta, cada alumno deberá realizar el siguiente análisis: ¿Si pudieras cambiar una pieza de tu computadora para que la prueba se pudiera mejorar, qué cambiarías? ¿Cuánto costaría el cambio? ¿Cuánta sería la mejora que este cambio da?}
    
    \begin{itemize}
        \item \textbf{Prueba de Tiempo de Respuesta:} 
        Para la prueba de Git, lo que se podria cambiar es aumentar la memoria RAM de 8 GB a 16 GB con otra memoria de 8 GB, con un costo de 661 pesos mexicanos en amazon.  Con esta mejora se espera reducir a la mitad el tiempo promedio de la prueba.
        \item \textbf{Prueba de Rendimiento:}
        Para la prueba de 7zip, lo que se podria cambiar es el procesador para aumentar los MIPS y dependiendo del procesador el costo seria de un rango de 5000 - 10000 pesos mexicanos.
    \end{itemize}
\end{enumerate}

\newpage
\section{Preguntas}
\begin{enumerate}[label=(\arabic{section}.\arabic{subsection}.\arabic{enumi})]
    \item ¿Cuál computadora tiene el mejor tiempo de ejecución? Comparada con la computadora con la peor medida de tiempo de ejecución, ¿por qué factor es mejor la computadora? Enuncia el resultado de la forma “El tiempo de ejecución de la computadora A es x veces que la computadora B”.
    
    \item ¿Cuál computadora tiene el mejor rendimiento? Comparada con la computadora con el peor rendimiento, ¿por qué factor es mejor la computadora? Enuncia el resultado de la forma “El rendimiento de la computadora A es x veces que la computadora B”.

    \item Considera todas las computadoras usadas como referencia; Para cada computadora, ¿cuál computadora tiene el mejor desempeño y cuál computadora tiene el peor desempeño?

    \item ¿Qué es el Socket AM4 y AM5? ¿Cuáles son sus diferencias y cuál se usa más hoy en día? ¿Cuáles son los sockets LGA 1200 y LGA 1151? ¿Cuáles son sus diferencias y cuál se usa más hoy en día?
    
    Los Socket AM4 y AM5 son zócalos de CPU, creados por AMD y diseñados para procesadores AMD. Sus diferencias estan entre que los Sockets AM5 brindan un mejor desempeño que los Sockets AM4 y tambien cambian de una estructura PGA a una LGA; actualmente se ocupan mas los Sockets AM5, esto porque los nuevos procesadores se diseñan parqa seguir encajando en esta estrcutura. Los Sockets LGA 1200 y 1151 son zócalos de CPU creados por Intel para los procesadores de la misma empresa. la mayor diferencia que hay entre los sockets es que el socket 1200 tiene 49 pines mas que el socket 1151 y que el socket 1200 birmda mejor manejo de energia que su antecesora; actualmente el Socket mas usado es el LGA 1200.

    \item ¿Por qué se considera que la GPU: Nvidia GTX 1080Ti es una de las mejores GPUs de todos los tiempos?
    
    La GPU Nvidia GTX 1080 Ti fue una mejora bastante grande respecto a la GPU insignia de la serie anterior, dando un mejor desempeño respecto a la serie 09 las cuales tambien tenian un gran desempeño. Ademas de que la 1080 Ti tsigue teniendo un desempeño bastante decente comparado con las GPU actuales. Esto sumado a que las GPU de las series 16 que le siguieron a la 1080 Ti fueron de lo mejor que a sacadop a la venta Nvidia.

    \item De entre los atributos de cada máquina comparada, ¿cuál máquina tiene el mejor disco duro? ¿Cuál tiene la mejor GPU? ¿Cuál tiene la mejor RAM? ¿Cuáles resultan determinantes en la pérdida o ganancia de desempeño en las pruebas realizadas?
\end{enumerate}

\end{document}
